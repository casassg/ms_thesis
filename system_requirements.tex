\chapter{System Requirements}

To focus my thesis and make it easy to evaluate, I present the requirements for my prototype. 

Disaster analysis systems need to be able to collect social media data for various events such as cyclones, floods, earthquakes, and hurricanes, whenever they occur using the live stream of data from social media platforms. The collection needs to be reliable and capable of keeping up with peak message rates whenever needed. The system needs to be running continuously and should be able to collect data for multiple events in parallel. There should be an interface to manage events collected and to start new event collections.

Due to social media streams not having static data schema, data should be saved without any modification in an unstructured data storage mechanism. This storage needs to support fast retrieval of data elements for a single event. In addition, this solution should allow keeping the data for long periods of time with low usage so that future analysis can be run retrospectively on all historical data collected. The data storage should have a reliable mechanism in place to avoid losing any data successfully stored.

Data should be accessible by event via an easy-to-use user interface. This interface should allow analysts to scan the data quickly and be able to explore in depth if needed. In addition, analysts should be able to time slice the data to better understand what was happening at different times of the collection. These characteristics paired with a good data visualization of the number of messages collected by hour should help analysts explore user interactions throughout the time line of the event. Being able to understand in depth what was happening during an event in a timely manner is important as it helps analysts study the behavior of members of the public responding to or talking about a disaster event.

Finally, the system should allow for interactive queries to be performed in a high-level language like SQL. These queries should be computed with low latency to enable fast data exploration. This feature can help researchers better understand the data they are working with, enabling them to explore the data set in multiple ways. This exploration could enable better insight into how people react to emergency response. At the same time, it could become a way for first responders to better understand a situation with flexible knowledge of what is happening around them via social media messages. Reducing latency could prove really useful in managing interactions with the public in real time.
