\chapter{Results}

After checking the new infrastructure and comparing it to the old version, I am going to explore now if I can answer the questions from my problem statement. 

First, in regards to maintainability, my proposed infrastructure simplifies the overall complexity of the system by outsourcing many parts to cloud services. Thanks to that, the system is not required to run on a large cluster of machines that Project EPIC has to maintain. The system goes from a set of six machines that are really specific to the infrastructure to four machines that run on a Kubernetes cluster.  In addition, my use of microservices separates system responsibilities into different components. This approach creates smaller services as compared to the monolithic components of the old infrastructure. Reducing the responsibility of each service makes the system easier to maintain. This is due to developers not having to understand the whole system at once and being able to work on single services individually. In addition, thanks to Kubernetes, developers do not need a system administrator to deploy their code; they just need to understand how containerization works. Once they have wrapped their microservice into a container, it can easily be deployed and plugged into all of the reliability and scalability features that Kubernetes provides. 

With respect to scalability, thanks to using BigQuery, the analysis pipeline scales better than before. This is due to this service being based on server-less cloud infrastructure. On the services side, with Kubernetes, services can scale horizontally using its horizontal auto-scaler. This is especially useful for the ingestion pipeline, as it allows for ingestion to adapt to demand. On the front-end, I can not even compare as both approaches are different. The new infrastructure though should be able to hold higher throughput thanks to the detachment of views and controllers. 

For reliability, Kubernetes provides abstractions to ensure the recovery of services automatically. Thanks to this and the microservice architecture, the system can be upgraded without bringing it down. There have been a few Kubernetes upgrades during development without much of a delay or error in the service provided. In addition, with Kubernetes rolling upgrades, the system first tests any new version of a service before switching incoming requests to it. In addition, the system switches from placing trust in Project EPIC being able to maintain hardware over time to a company that is focused on doing just that. 

Finally, this infrastructure has proved already that it can enable multiple extensions being developed in parallel. It is the case for the mentions API discussed above. The developer of that service was able to develop the extension fully on its own with ease. This proves that group work can be done on top of the current infrastructure in parallel, enabling easier collaboration between different researchers at Project EPIC, as each collaborator can work on their extension for the project with ease.
