\chapter{Conclusion}
\OnePageChapter

Cloud infrastructures are changing the software engineering world by providing new alternatives to traditional architecture. This allows for new software architecture practices to be born. It is a change of perspective from traditional architectures. The more interesting part is the growth of server-less infrastructures, which are an evolution from container orchestrated infrastructures. Server-less services on the cloud will change how systems are designed in the future. New software engineering practices will be born from this new abstraction layer to provide future developers with new tools to design and build good software systems.

My research proves that there is a lot of potential within this world for software engineers. Research needs to be done in order to check if more use cases can benefit from the use of these tools as I did in my approach. 

In conclusion, I have been able to show that it is possible to reduce the maintenance cost of Project EPIC’s infrastructure, while replicating most of its features, in a cloud environment. Proving that cloud services can work in our favor to reduce costs, increasing reliability as well as scalability. While at the same time, it helps to reduce friction between collaborators to extend the system thanks to the microservice approach and to the reduction of dependencies within the system.
