\chapter{Discussion}

After my work during my undergraduate thesis to explore containerization and container orchestrated systems, I was able to understand the advantages of using cloud infrastructures. With my thesis, I pointed at all the advantages of using microservices for collecting Twitter data. I also suggested building upon the prototype I did to create a new infrastructure for Project EPIC. This is fully realized in this current work. Here, I used the work done in my undergraduate thesis to expand from the collection pipeline and also supply the analysis interface. Thanks to my experience I have obtained through these years of research at Project EPIC, I have been able to design and fix errors I tackled in the previous iteration. In addition, I have been able to understand better the requirements from external and internal collaborators for this system.

I have also learned how to manage a team of students to develop this project. I believe it is a great experience and it has been really great to work alongside other people. I feel like some times research groups tend to work too separately and collaboration gets relegated as a small part. The results of this current work are such thanks to the active collaboration and management from the group of students I worked with. I would strongly encourage any future developments to incorporate a lead and more developers as this project did. 

Another topic of discussion is whether storing this data in external servers, like Google’s in our case, is a good ethical decision. Relying on external actors for research can lead to impartial results in favor of those that provide them. Given the latest scandals for bad usage of data from various companies in the tech industry, I believe that it is the duty for non-profit research to investigate and keep these companies accountable. It is in society’s best interest to have research on how private companies use our data to benefit themselves. This research can only come from impartial external parties, like universities. Relying on them to conduct research can prove to influence the process of accountability from universities. I believe it is important to power software engineering research in universities for this same reason. If universities fall behind from the industry, then we will only have ad-driven research which may not be the best benefit for society as a whole as we have seen in the latest years.
